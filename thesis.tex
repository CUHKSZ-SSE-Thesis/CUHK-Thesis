%!TEX program=xelatex
\documentclass[12pt,a4paper,twoside]{report}
%CHANGLOG
% 1st Version is revised by ZHENG Fan (fzheng@link.cuhk.edu.hk) according to
% CUHK Grad. School's requirements on thesis format. Check:
%   https://www.gradsch.cuhk.edu.hk/pgstudent/gsinfo/research/Chapter%206.html
% 2nd Version is revised by ZHOU Yuming (121050081@link.cuhk.edu.cn) and AN Zihang (121090001@link.cuhk.edu.cn)
% with modifications on formats required by CUHKSZ made and comments added.

%=======================================================

% For Chinese and fonts
% CUHKSZ has no limitations on font-types and no requirement for whether traditional or simplified Chinese should be used
\usepackage{fontspec}
\usepackage[scheme=plain]{ctex}
% \setCJKmainfont{SimSun} % 可自行更改为系统中文字体;SimSun即Windows宋体。

%=======================================================

\input{format/style}
\input{format/symbols}

%=======================================================

% margins and spacing required by CUHKSZ
% 4 cm margin for binding edge, 2.5 cm for outer edge:
\usepackage[inner=4cm,outer=2.5cm]{geometry}
\usepackage{setspace}
% 1.5 line spacing for normal texts:
\setstretch{1.5}
% avoid too much space automatically padded between paragraphs to fill a page:
\raggedbottom

%=======================================================

% personally imported packages and customization
\usepackage[mono=false]{libertine} % I personally favour this font
\usepackage{subfig}
\usepackage{tikz}
\usepackage{pgfplots}
\graphicspath{{figures/}}
\usepackage[authoryear,square]{natbib} % citation style management
\def\cite{\citep}
% CUHKSZ has no limitations on bibliography style, use your favor.
%\bibliographystyle{ieeetr} % for 'nubmers' citation style
\bibliographystyle{apalike} % for 'authoryear' citation style
\usepackage[colorlinks=true,linkcolor=blue,citecolor=magenta]{hyperref}

%=======================================================

\begin{document}


\thesistitle{TITLE OF YOUR THEIS OR DISSERTATION}% title of your thesis
\thesistitlezh{香港中文大学深圳学位论文标题} % CUHKSZ 对简繁体不作要求

\authorname{LI Tiezhu} % English name
\authornamezh{李铁柱} % Chinese name


\degree{Master of Philosophy} % degree


\programme{Computer and Information Engineering} % programme
\institution{The Chinese University of Hong Kong, Shenzhen} % institution
\submitdate{August 1926} % submitdate
\committee{% commitee list
	Professor ZHANG San (Chair)\\
	Professor LI Si (Thesis Supervisor)\\
	Professor WANG Wu (Thesis Co-supervisor)\\
	Professor ZHAO Liu (Committee Member)\\
	Professor JIN Guo (Examiner from CUHK)\\
	Professor WHO Ever (External Examiner)
}
\coverpage

%=======================================================

\pagenumbering{roman}
\abstractpage

%=======================================================

\acknowledgementpage

%=======================================================

\tableofcontents
\listoffigures
\listoftables
\input{sections/symbols-and-acronyms.tex}

%=======================================================

\newpage
\setcounter{page}{0}
\pagenumbering{arabic}
\pagestyle{headings}

%=======================================================

%insert the chapters
\input{sections/introduction.tex}
\input{sections/chapter-one.tex}
\input{sections/chapter-two.tex}
\input{sections/conclusion.tex}

%=======================================================

\appendix
\input{sections/proof.tex}
\input{sections/publications.tex}

%=======================================================

\newpage
% Double or one-and-a-half line spacing should be used, except for quotations, footnotes, references and captions, which may be single-spaced.
\setstretch{1}
\bibliography{database}

\end{document}
























% 加油!
% 不要放弃!
% 咱们龙岗人最会卷了!
% 洋文和论文多打磨打磨总会出成绩的!